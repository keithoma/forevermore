\documentclass[refman]{article}
\usepackage[utf8]{inputenc}
\usepackage[english]{babel}


%

\usepackage{colortbl}
\usepackage{epigraph}
\usepackage{fancyhdr}
\usepackage{graphicx}
\usepackage{hhline}
\usepackage{biblatex}

\usepackage[procnames]{listings}
\usepackage{longtable}
\usepackage{tikz}
\usepackage{subcaption}
\usepackage{xcolor}
\usepackage{wrapfig}
\usepackage[nottoc,numbib]{tocbibind}
%
%\usepackage[T1]{fontenc}
\usepackage{lmodern}

\usepackage{amsfonts}
\usepackage{amsmath}
\usepackage{amsthm}
\usepackage{mathtools}

\usepackage{geometry}
 \geometry{
 a4paper,
 left=30mm,
 top=30mm,
 }
\pagestyle{fancy}

\newcommand{\idx}{\text{idx}}

\DeclarePairedDelimiter\ceil{\lceil}{\rceil}
\DeclarePairedDelimiter\floor{\lfloor}{\rfloor}

\theoremstyle{definition}
\newtheorem*{formula}{Formula}

\usepackage[document]{ragged2e}

\begin{document}
	\begin{titlepage}
	\centering
	\includegraphics[width=0.15\textwidth]{graphic/huberlin_logo}\par\vspace{1cm}
	{\scshape\LARGE Humboldt University of Berlin \par}
	\vspace{1cm}
	{\scshape\Large Projektpraktikum \par}
	\vspace{1.5cm}
	{\huge\bfseries Numerical Approach to the Laplace Problem \par}
	\vspace{2cm}
	{\Large\itshape Christian Parpart \& Kei Thoma \par}
	\vfill

	\vfill

% Bottom of the page
	{\large \today\par}
\end{titlepage}

\tableofcontents
	\newpage
    \justify
    \section{Introduction}

This protocol discusses a numerical approach to solve a Laplace equation which for an \(u \in C^2(\mathbb{R^d}, \mathbb{R})\) is defined as
\begin{align*}
	\Delta u = \sum^{d}_{l = 1} \frac{\partial^2 u}{\partial x_l^2} \text{.}
\end{align*}
In particular, we will limit ourselves on the domain \( \Omega := (0, 1)^d \subset \mathbb{R^d} \) with the equations\footnote{\(\partial \Omega\) denotes the boundary of the domain.}
\begin{align*}
	- \Delta u &= f \hspace{0.5cm} \text{on } \Omega \text{, and} \\
	u &= 0 \hspace{0.5cm} \text{on } \partial \Omega \text{.}
\end{align*}
Given the function \(f\), how can one compute \(u\)? First, we must define a proper discretization of the domain and a convenient linear order for the grid points. This allows us to encode the values of the functions \(u\) and \(f\) for each grid point into a vector \(\hat{u}\) and \(b\). The notation for the vector \(\hat{u}\) is not just to differentiate from the function \(u\) we want to compute, but also to stress that the solution will be an approximation.

It turns out that we can actually construct a fairly simple linear transformation which maps the vector \(\hat{u}\) to \(b\). In other words, we will see that there is a matrix \(A\) such that
\begin{align*}
	A \hat{u} = b \text{.}
\end{align*}

Furthermore, this particular matrix \(A\), while large in size for almost all pratical applications, has a useful structure which we will exploit to reduce computation time to an acceptable length.

Lastly, as this protocol is meant to be preparatory piece, it will not present a method to solve the above linear equation for \(\hat{u}\).
    \section{Theoretical Considerations}

Every journey must begin with a single step. In our case, we consider a one dimensional domain, i.e.
\begin{align*}
    \Omega = (0, 1)
\end{align*}
before we generalize to multiple dimension. It should be noted that the case for the first dimension is particularly simple because instead of any partial derivatives, we just have the equation
\begin{align*}
    - u'' = f \text{.}
\end{align*}
Indeed, the multidimensional case require more theoretical work, but in essence, the core idea to construct matrices stays the same. Understanding the first dimension will therefore be our stepping stone for a general method to solve the Laplace equation.

\subsection{Discretization of Lines}

It might be tempting for some to solve the given differential equation for every point on the interval \([0, 1] \subset \mathbb{R}\), but with a numerical approach the price for such computation of infinitely many points would exceed every budget, and therefore, we must instead limit our aspirations to a finite amount of grid points. In particular, we will arrange \(n\) number of points on the said interval which are exactly \(h\) distance apart from the next. Hence, we have
\begin{align*}
	h := \frac{1}{n} \text{,}
\end{align*}
and moreover, we will number each point \(p_i \in [0, 1]\) where \(0 \leq i \leq n\)  on the interval. This gives us
\begin{align*}
	p_i = \frac{j}{n} \text{.}
\end{align*}

The Laplace equation will only be evaluated on these points \(p_i\). This means that a finer discretization will give a better approximated result, but the computation time will also increase.

Here, we have only described the discretization used for the first dimension. We will return to this topic later when we discuss the arrangement of grid points in multidimensional space.

\subsection{Bend it like Taylor}

Before we even consider partial derivatives, it is critical to introduce a method to evaluate derivatives numerically. Taylor's theorem is a powerful tool with applications in many fields and branches, but in numerical analysis, we use the Taylor polynomial to compute the approximations for the derivatives of smooth functions.

Let \([0, 1] \subset \mathbb{R}\) be a interval with \(n\) number of grid points, \(f \in C^{\infty}([a, b], \mathbb{R})\) a function, and we will denote with \(h \in \mathbb{R}\), \(h > 0\) the increment of the approximations. Given an \(p \in (a, b)\), we define \(p_{+} := p + h\) and \(p_{-} := p - h\).

By Taylor's theorem \cite{H.Amann} we have
\begin{align}
    f(p_{+}) &= \sum^{\infty}_{n = 0} \frac{f^{(n)}(p)}{n!} h^n = f(p) + f'(p)h + \frac{f''(p)}{2}h^2 + \dots \label{eq:1}\\
    f(p_{-}) &= \sum^{\infty}_{n = 0} \frac{f^{(n)}(p)}{n!} (-h)^n = f(p) - f'(p)h + \frac{f''(p)}{2}h^2 + \dots \label{eq:2}
\end{align}
Reformulate this equation and define the numerical approximation of the first derivative to be
\begin{align*}
    (D^{(1)}_h f) (p) := \frac{f(p_{+}) - f(p)}{h} = f'(p) + \sum^{\infty}_{n = 2} \frac{f^{(n)} (p)}{n!}h^{n-1} \text{.}
\end{align*}
Adding (\ref{eq:1}) and (\ref{eq:2}) together and reformulating gives us
\begin{align*}
    (D^{(2)}_h f)(p) := \frac{f(p_{+}) - 2f(p) + f(p_{-})}{h^2} = f''(p) + \frac{f^{(4)}(p)h^2}{3 \cdot 4} + \dots \text{.}
\end{align*}
\(D^{(2)}_h\) is the numerical approximation of the second derivative.

As the remainder of the Taylor polynomial approaches \(0\) for \(h \rightarrow 0\), the approximations uniformly converge to the analytic derivatives.

\subsection{The Chosen Increment}

Now we are equipped to solve the stated problem for the first dimension. There is, however, one very important thing to mention. The increment \(h\) which we will use to approximate the derivatives will set to be exactly the distance between each neighboring grid points. This trick proves to be very useful when we construct the matrices.
    \subsection{First Dimension}
As the behavior on the boarder is given, we immediately know that
\begin{align*}
    u(p_0) = u(p_n) = 0 \text{.}
\end{align*}
Moreover, since we are in the first dimension, we have \(u \in C^2(\mathbb{R}, \mathbb{R})\), therefore the second derivative of \(u\) is nothing complicated. We will approximate the computation of the second derivative with the Taylor polynomial and we have
\begin{align*}
    f(p_i) = -u''(p_i) \approx \frac{-u(p_i + h) + 2u(p_i) - u(p_i - h)}{h^2} \text{,}
\end{align*}
and as \(h\) is not only the increment of the approximation, but also the distance between each point. For notational convenience, denote \(u_i := u(p_i)\) and \(f_i := f(p_i)\)  we get
\begin{align*}
    f_i \approx \frac{1}{h^2} (-u_{i - 1} + 2u_i - u_{i + i}) \text{.}
\end{align*}
Let \(b \in \mathbb{R}^{n-1}\) be a vector which encodes for \(1 \leq i \leq n-1\) the values of \(f(p_i) =: f_i\). Then we have
\begin{align*}
    b =
    \begin{pmatrix}
        f_1 \\
        f_2 \\
        f_3 \\
        \vdots \\
        f_{n-1}
    \end{pmatrix}
    \approx
    \frac{1}{h^2}
    \begin{pmatrix}
        -u_0 + 2u_1 -u_2 \\
        -u_1 + 2u_2 -u_3 \\
        -u_2 + 2u_3 -u_4 \\
        \vdots \\
        -u_{n-2} + 2u_{n-1} -u_{n}
    \end{pmatrix}
    =
    \frac{1}{h^2}
    \begin{pmatrix}
        2u_1 -u_2 \\
        -u_1 + 2u_2 -u_3 \\
        -u_2 + 2u_3 -u_4 \\
        \vdots \\
        -u_{n-2} + 2u_{n-1}
    \end{pmatrix} \text{.}
\end{align*}
Above, we have used the approximation of the second derivative and the fact that \(u_0 = u_n = 0\).

From this representation of \(b\), we can define the matrix \(A_1 \in \mathbb{R}^{(n-1) \times (n-1)}\) for the linear transformation \(A_1 \hat{u} = b\). We have
\begin{align*}
    \frac{1}{h^2}
    \underbrace{
    \begin{pmatrix}
        2  & -1 &  0 & \dots & 0 \\
        -1 &  2 & -1 & \dots & 0 \\
        0  & -1 &  2 & \dots & 0 \\
        \vdots & & & \ddots & \vdots \\
        0 & \dots & 0 & -1 & 2
    \end{pmatrix}
    }_{=: A_1}
    \underbrace{
    \begin{pmatrix}
        u_1 \\ u_2 \\ u_3 \\ \vdots \\ u_{n-1}
    \end{pmatrix}
    }_{=: \hat{u}}
    =
    \frac{1}{h^2}
    \begin{pmatrix}
        2u_1 - u2 \\
        -u_1 + 2u_2 + -u_3 \\
        -u_2 + 2 u_3 - u_4 \\
        \vdots \\
        -u_{N-2} + 2 u_N
    \end{pmatrix}
    \approx
    \underbrace{
    \begin{pmatrix}
        f_1 \\
        f_2 \\
        f_3 \\
        \vdots \\
        f_{n-1}
    \end{pmatrix}
    }_{=b}
\end{align*}
    \subsection{Discretization of the Space}

In a nutshell, we have constructed the matrix \(A_1\) by exploiting the fact that the increment for the approximation is the same as the distance between two grid points which allows us to express the derivative at a certain point with the neighboring values, i.e.
\[
f(p_i) = -u''(p_i) = \frac{1}{h^2} (-u_{i - 1} + 2 u_i - u_{i + 1}) \text{.}
\]
We will use essentially the same idea for higher dimensions. However, instead of looking into just one direction (that is two neighboring points on an axis), we will be looking at two and three.

Let \([0,1]^{d}\) with \(d \in \{1, 2, 3\}\) the domain for the given problem. Again, on the boundaries \(u\) is \(0\). The discretization of this domain arises naturally from the one dimensional case. Each axis is equipped with \(n \in \mathbb{N}\) number of Cartesian grid such that there is a total of \( N := (n-1)^d\) points to compute. To proceed the discretization alone is not enough however. We need a way to order these points in the domain. For this endeavor, we start with a function that maps the Cartesian coordinates to an positive integer \(1 \leq i \leq N\).

\begin{formula}
    For the conversion from coordinates along the axis to the linear numbering of gird points and vice versa, we define
    \begin{align*}
        \idx_{d} &: \{1, \dots, n - 1\}^d \rightarrow \{1, \dots, N\} \text{, and} \\
        \idx_{d}^{-1} &: \{1, \dots, N\} \rightarrow \{1, \dots, n - 1\}^d \text{.}
    \end{align*}
    The exact way of computation is intuitive, yet tedious to formulate symbolically.

    \noindent \textit{Dimension 1}
    \begin{align*}
        \idx_{d = 1} (j_1) & = j_1 \\
        \idx_{d = 1}^{-1} (j_1) & = j_1 \\
    \end{align*}
    \textit{Dimension 2}
    \begin{align*}
        \idx_{d = 2} (j_1, j_2) & = (j_1 - 1) (n - 1) + j_2 \\
        \idx_{d = 2}^{-1} (N) & = (j_1, j_2) \text{, where}
        \begin{cases}
            j_1 = \ceil*{\frac{N}{n - 1}} \\
            j_2 = N - (j_1 - 1) (n - 1)
        \end{cases}
    \end{align*}
    \textit{Dimension 3}
    \begin{align*}
        \idx_{d = 3} (j_1, j_2, j_3) & = (j_1 - 1) (n - 1)^2 + (j_2 - 1) (n - 1) + j_3 \\
        \idx_{d = 3}^{-1} (N) &= (j_1, j_2, j_3) \text{,} \\
        \text{where}&
        \begin{cases}
            j_1 = \ceil*{\frac{N}{(n - 1)^2}} \\
            \\
            j_2 = \ceil*{\frac{N - (j_1 - 1) (n - 1)^2}{n - 1}} \\
            \\
            j_3 = N - (j_2 - 1) (n - 1) - (j_1 - 1) (n - 1)^2
        \end{cases}
    \end{align*}

    \textit{Derivation} \hspace{0.1cm} The formula for \(d = 1\) should be clear. For \(d = 2\), given two coordinates \(j_1\) and \(j_2\), then one has \(j_1 - 1\) times of filled columns which has \(n - 1\) elements, thus we have \(N = (j_1 - 1) (n - 1) + j_2\). The inverse is slightly more difficult. To compute \(j_1\), we need to consider how many columns are filled by \(N\). This is done by \(\ceil*{\frac{N}{n-1}}\). Subtracting \((j_1 - 1)(n - 1)\) from \(N\) we get \(j_2\). The same ideas applies for \(d=3\).
    \begin{flushright}
        \(\bigtriangleup\)
    \end{flushright}
\end{formula}

Now, we have a bijective function which maps Cartesian coordinates to a positive integer. The linear ordering of the grid points arises naturally by ordering through the integer numbering asigned to each point, i.e. for a points \(p\) and \(q\)  with \(j_p\) and \(j_q\) as tuples of coordinates
\begin{align*}
    p < q :\iff idx(j_p) < idx(j_q)
\end{align*}
Intuitively, this ordering relation can be understood as a lexicographic ordering of the Cartesian coordinates.


% number them correctly
    As the behaviour on the boarder is given, we immediately know that
\begin{align*}
    u(x_0) = u(x_N) = 0 \text{.}
\end{align*}
Moreover, since we are in the first dimension, we have \(u \in C^2(\mathbb{R}, \mathbb{R})\), therefore the second derivative of \(u\) is nothing complicated. We will approximate the computation of the second derivative with the Taylor polynomial and we have
\begin{align*}
    f(x_i) = -u''(x_i) \approx \frac{-u(x_i + h) + 2u(x) - u(x_i - h)}{h^2} \text{,}
\end{align*}
and as \(h\) is not only the increment of the approximation, but also the distance between each point, we get
\begin{align*}
    h^2 f_i \approx -u_{i - 1} + 2u_i - u_{i + i} \text{.}
\end{align*}
If we express this relation for all \(1 \leq i \leq N - 1\) as a linear transformation, we have
\begin{align*}
    \underbrace{
    \begin{pmatrix}
        2  & -1 &  0 & \dots & 0 \\
        -1 &  2 & -1 & \dots & 0 \\
        0  & -1 &  2 & \dots & 0 \\
        \vdots & & & \ddots & \vdots \\
        0 & \dots & 0 & -1 & 2
    \end{pmatrix}
    }_{=: A_1}
    \begin{pmatrix}
        u_1 \\ u_2 \\ u_3 \\ \vdots \\ u_{N-1}
    \end{pmatrix}
    =
    \begin{pmatrix}
        2u_1 - u2 \\
        -u_1 + 2u_2 + -u_3 \\
        -u_2 + 2 u_3 - u_4 \\
        \vdots \\
        -u_{N-2} + 2 u_N
    \end{pmatrix}
\end{align*}
For the resulting vector on the right side, we can subtract from the first and last entry \(u_0\) and \(u_N\) respectively since they are both \(0\). We have
\begin{align*}
    \begin{pmatrix}
        2u_1 - u2 \\
        -u_1 + 2u_2 + -u_3 \\
        -u_2 + 2 u_3 - u_4 \\
        \vdots \\
        -u_{N-2} + 2 u_{N - 1}
    \end{pmatrix}
    % ceck here
    =
    \begin{pmatrix}
        -u_0 + 2u_1 - u2 \\
        -u_1 + 2u_2 + -u_3 \\
        -u_2 + 2 u_3 - u_4 \\
        \vdots \\
        -u_{N-2} + 2 u_{N - 1} - u_N
    \end{pmatrix}
    =
    \begin{pmatrix}
        2u_1 - u2 \\
        -u_1 + 2u_2 + -u_3 \\
        -u_2 + 2 u_3 - u_4 \\
        \vdots \\
        -u_{N-2} + 2 u_N
    \end{pmatrix}
\end{align*}
    \subsection{Saving Memory}
The aforementioned matrix \(A_d\) increases in size quickly for large \(n\) especially for \(d=3\), however, its entries are mostly zero. The SciPy library offers a matrix class which condenses a matrix by taking the nonzero entries and its indices. These matrix are called \textit{sparse}. We want to compare the memory usage for a dense and sparse matrix. This, of course, depends on the data type of the entries, the exact way how the class allocates memory et cetera, but we can at least formulate a way to count the nonzero entries for sparse matrices.

\begin{formula}
    Let \(d \in \{1, 2, 3\}\) be the number of the space dimension and \(n \in \mathbb{N}\) the number of grid points on each axis \([0, 1]\) excluding the start point, but including the end point (i.e. since the given problem has the value \(0\) at the boundary, we will effectively evaluate \((n - 1)^d\) points). Then, we have the formula for all entries of dense and the nonzero entries for sparse matrices
    \begin{align*}
        f^{(d = 1)}_{\text{dense}} (n) = (n - 1)^2 \hspace{0.5cm}&
        f^{(d = 1)}_{\text{sparse}} (n) = 3n - 5 \\
        %
        f^{(d = 2)}_{\text{dense}} (n) = (n - 1)^4 \hspace{0.5cm}&
        f^{(d = 2)}_{\text{sparse}} (n) = 5 (n - 1)^2 - 4 (n - 1) \\
        %
        f^{(d = 3)}_{\text{dense}} (n) = (n - 1)^6 \hspace{0.5cm}&
        f^{(d = 3)}_{\text{sparse}} (n) = 7 (n - 1)^3 - 6 (n - 1)^2 \text{.}
    \end{align*}

    \textit{Derivation} \hspace{0.1cm} The memory usage for dense matrix are precisely the number of elements of the matrix. Hence,
    \begin{align*}
        f^{(d)}_{\text{dense}} (n) =
        (n - 1)^d \cdot (n - 1)^d = (n - 1)^{2d}
    \end{align*}
    is the memory usage of the dense matrix.

    For sparse matrices, we consider each dimension separately. If \(d = 1\), then the diagonal containing \(2\) has \(n - 1\) elements. Additionally, the upper and lower diagonal shifted by one containing \(-1\) has each \(n - 2\) elements. Thus, we have
    \begin{align*}
        f^{(d = 1)}_{\text{sparse}} (n) & = (n - 1) + 2 (n - 2) \\
        & = 3n - 5 \text{.} 
    \end{align*}

    On the other hand, if \(d = 2\), then the matrix \(A_2\) contains \(n - 1\) times \(A_1\) with each having \(3n - 5\) elements. \(A_2\) also contains total of \(2 (n - 2)\) diagonal matrices with \(-1\) as entries which each again has \(n - 1\) elements. Summing all together, we have
    \begin{align*}
        f^{(d = 2)}_{\text{sparse}} (n) & = (n - 1) (3n - 5) + 2 (n - 2) (n - 1) \\
        & = 5 (n - 1)^2 - 4 (n - 1) \text{.}
    \end{align*}

    Lastly, if \(d = 3\), then the matrix \(A_3\) again contains \(n - 1\) times \(A2\) and \(2 (n - 2)\) diagonal matrices of the size \((n - 1)^2\). This gives us
    \begin{align*}
        f^{(d = 3)}_{\text{sparse}} (n) & = (n - 1) (5 (n - 1)^2 - 4 (n - 1)) + (2 (n - 2)) (n - 1)^2 \\
        & = 7 (n - 1)^3 - 6 (n - 1)^2 \text{.}
    \end{align*}
    \begin{flushright}
        \(\bigtriangleup\)
    \end{flushright}
\end{formula}
\begin{figure}[h]
    \includegraphics[width=\linewidth]{graphic/memory.png}
    \caption{The formula for the number of entries visualized.}
    \label{fig:plot}
\end{figure}
Perhaps the above formula might not be too exciting for some. For those, we have written a Python module \texttt{count\_entries.py} which visualizes the above formula for \(2 \leq n \leq 10000\) (see figure \ref{fig:plot}). One may change the \texttt{MAX\_N} variable to draw the graph for a different upper bound of \(n\).

\section{Conclusion}
The presented method to tackle the Laplace equation for three dimensions can further generalized to any number of finite dimensions. This is intuitive, but notationally intensive.

Furthermore, we must talk about the elephant in the room. We have essentially reduced a problem of partial differential equations to a problem of linear transformation, but how are we going to numerically solve the system for \(\hat{u}\)? In the next protocol, we will return to this issue and strengthen our understanding about the Laplace equation by actually solving the matrix \(A_d\).
\medskip
\begin{thebibliography}{9}
    \bibitem{H.Amann}
    Herbert Amann, Joachim Escher
    \textit{Analysis I} (German). 
    3rd edition, Birkhaeuser Verlag, Berlin, 2006.
\end{thebibliography}

    \newpage
    \epigraph{"Do not try and solve the system of linear equation, that's impossible. Instead, only try to realize the truth ... there is no system of linear equation."}{Unknown}
\end{document}