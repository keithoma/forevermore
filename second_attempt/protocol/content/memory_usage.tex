\begin{formula}
    Let \(d \in {1, 2, 3}\) be the number of the space dimension and \(n \in \mathbb{N}\) the number of grid points on each axis \([0, 1]\) excluding the start point, but including the end point (i.e. since the given problem has the value \(0\) at the boundary, we will effectivly evaluate \((n - 1)^d\) points). Then, we have the formula for the memory usage of dense and sparse matrices
    \begin{align*}
        f^{(d = 1)}_{\text{dense}} (n) = (n - 1)^2 \hspace{0.5cm}&
        f^{(d = 1)}_{\text{sparse}} (n) = 3n - 5 \\
        %
        f^{(d = 2)}_{\text{dense}} (n) = (n - 1)^4 \hspace{0.5cm}&
        f^{(d = 2)}_{\text{sparse}} (n) = 5 (n - 1)^2 - 4 (n - 1) \\
        %
        f^{(d = 3)}_{\text{dense}} (n) = (n - 1)^6 \hspace{0.5cm}&
        f^{(d = 3)}_{\text{sparse}} (n) = 7 (n - 1)^3 - 6 (n - 1)^2 \text{.}
    \end{align*}
    \textit{Derivation} \hspace{0.1cm} The memory usage for dense matrix are precisely the number of elements of the matrix. Hence,
    \begin{align*}
        f^{(d)}_{\text{dense}} (n) =
        (n - 1)^d \cdot (n - 1)^d = (n - 1)^{2d}
    \end{align*}
    is the memory usage of the dense matrix.

    For sparse matrices, we consider each dimension separately. If \(d = 1\), then the diagonal containing \(2\) has \(n - 1\) elements. Additionally, the upper and lower diagonal shifted by one containing \(-1\) has each \(n - 2\) elements. Thus, we have
    \begin{align*}
        f^{(d = 1)}_{\text{sparse}} (n) & = (n - 1) + 2 (n - 2) \\
        & = 3n - 5 \text{.} 
    \end{align*}

    On the other hand, if \(d = 2\), then the matrix \(A_2\) contains \(n - 1\) times \(A_1\) with each having \(3n - 5\) elements. \(A_2\) also contains total of \(2 (n - 2)\) diagonal matrices with \(-1\) as entries which each again has \(n - 1\) elements. Summing all together, we have
    \begin{align*}
        f^{(d = 2)}_{\text{sparse}} (n) & = (n - 1) (3n - 5) + 2 (n - 2) (n - 1) \\
        & = 5 (n - 1)^2 - 4 (n - 1) \text{.}
    \end{align*}

    Lastly, if \(d = 3\), then the matrix \(A_3\) again contains \(n - 1\) times \(A2\) and \(2 (n - 2)\) diagonal matrices of the size \((n - 1)^2\). This gives us
    \begin{align*}
        f^{(d = 3)}_{\text{sparse}} (n) & = (n - 1) (5 (n - 1)^2 - 4 (n - 1)) + (2 (n - 2)) (n - 1)^2 \\
        & = 7 (n - 1)^3 - 6 (n - 1)^2 \text{.}
    \end{align*}
    \begin{flushright}
        \(\bigtriangleup\)
    \end{flushright}
\end{formula}