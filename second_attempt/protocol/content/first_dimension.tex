\subsection{First Dimension}
As the behavior on the boarder is given, we immediately know that
\begin{align*}
    u(p_0) = u(p_n) = 0 \text{.}
\end{align*}
Moreover, since we are in the first dimension, we have \(u \in C^2(\mathbb{R}, \mathbb{R})\), therefore the second derivative of \(u\) is nothing complicated. We will approximate the computation of the second derivative with the Taylor polynomial and we have
\begin{align*}
    f(p_i) = -u''(p_i) \approx \frac{-u(p_i + h) + 2u(p_i) - u(p_i - h)}{h^2} \text{,}
\end{align*}
and as \(h\) is not only the increment of the approximation, but also the distance between each point. For notational convenience, denote \(u_i := u(p_i)\) and \(f_i := f(p_i)\)  we get
\begin{align*}
    f_i \approx \frac{1}{h^2} (-u_{i - 1} + 2u_i - u_{i + i}) \text{.}
\end{align*}
Let \(b \in \mathbb{R}^{n-1}\) be a vector which encodes for \(1 \leq i \leq n-1\) the values of \(f(p_i) =: f_i\). Then we have
\begin{align*}
    b =
    \begin{pmatrix}
        f_1 \\
        f_2 \\
        f_3 \\
        \vdots \\
        f_{n-1}
    \end{pmatrix}
    \approx
    \frac{1}{h^2}
    \begin{pmatrix}
        -u_0 + 2u_1 -u_2 \\
        -u_1 + 2u_2 -u_3 \\
        -u_2 + 2u_3 -u_4 \\
        \vdots \\
        -u_{n-2} + 2u_{n-1} -u_{n}
    \end{pmatrix}
    =
    \frac{1}{h^2}
    \begin{pmatrix}
        2u_1 -u_2 \\
        -u_1 + 2u_2 -u_3 \\
        -u_2 + 2u_3 -u_4 \\
        \vdots \\
        -u_{n-2} + 2u_{n-1}
    \end{pmatrix} \text{.}
\end{align*}
Above, we have used the approximation of the second derivative and the fact that \(u_0 = u_n = 0\).

From this representation of \(b\), we can define the matrix \(A_1 \in \mathbb{R}^{(n-1) \times (n-1)}\) for the linear transformation \(A_1 \hat{u} = b\). We have
\begin{align*}
    \frac{1}{h^2}
    \underbrace{
    \begin{pmatrix}
        2  & -1 &  0 & \dots & 0 \\
        -1 &  2 & -1 & \dots & 0 \\
        0  & -1 &  2 & \dots & 0 \\
        \vdots & & & \ddots & \vdots \\
        0 & \dots & 0 & -1 & 2
    \end{pmatrix}
    }_{=: A_1}
    \underbrace{
    \begin{pmatrix}
        u_1 \\ u_2 \\ u_3 \\ \vdots \\ u_{n-1}
    \end{pmatrix}
    }_{=: \hat{u}}
    =
    \frac{1}{h^2}
    \begin{pmatrix}
        2u_1 - u2 \\
        -u_1 + 2u_2 + -u_3 \\
        -u_2 + 2 u_3 - u_4 \\
        \vdots \\
        -u_{N-2} + 2 u_N
    \end{pmatrix}
    \approx
    \underbrace{
    \begin{pmatrix}
        f_1 \\
        f_2 \\
        f_3 \\
        \vdots \\
        f_{n-1}
    \end{pmatrix}
    }_{=b}
\end{align*}