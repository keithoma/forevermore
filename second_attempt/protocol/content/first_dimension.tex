\subsection{First Dimension}
\vspace*{\fill}
    \epigraph{Nobody can be told what the matrix is, you have to see it for yourself.}{\emph{Morpheus}}

As the behavior on the boarder is given, we immediately know that
\begin{align*}
    u(x_0) = u(x_N) = 0 \text{.}
\end{align*}
Moreover, since we are in the first dimension, we have \(u \in C^2(\mathbb{R}, \mathbb{R})\), therefore the second derivative of \(u\) is nothing complicated. We will approximate the computation of the second derivative with the Taylor polynomial and we have
\begin{align*}
    f(x_i) = -u''(x_i) \approx \frac{-u(x_i + h) + 2u(x) - u(x_i - h)}{h^2} \text{,}
\end{align*}
and as \(h\) is not only the increment of the approximation, but also the distance between each point, we get
\begin{align*}
    h^2 f_i \approx -u_{i - 1} + 2u_i - u_{i + i} \text{.}
\end{align*}
If we express this relation for all \(1 \leq i \leq N - 1\) as a linear transformation, we have
\begin{align*}
    \underbrace{
    \begin{pmatrix}
        2  & -1 &  0 & \dots & 0 \\
        -1 &  2 & -1 & \dots & 0 \\
        0  & -1 &  2 & \dots & 0 \\
        \vdots & & & \ddots & \vdots \\
        0 & \dots & 0 & -1 & 2
    \end{pmatrix}
    }_{=: A_1}
    \begin{pmatrix}
        u_1 \\ u_2 \\ u_3 \\ \vdots \\ u_{N-1}
    \end{pmatrix}
    =
    \begin{pmatrix}
        2u_1 - u2 \\
        -u_1 + 2u_2 + -u_3 \\
        -u_2 + 2 u_3 - u_4 \\
        \vdots \\
        -u_{N-2} + 2 u_N
    \end{pmatrix}
\end{align*}
For the resulting vector on the right side, we can subtract from the first and last entry \(u_0\) and \(u_N\) respectively since they are both \(0\). We have
\begin{align*}
    \begin{pmatrix}
        2u_1 - u2 \\
        -u_1 + 2u_2 + -u_3 \\
        -u_2 + 2 u_3 - u_4 \\
        \vdots \\
        -u_{N-2} + 2 u_{N - 1}
    \end{pmatrix}
    % ceck here
    =
    \begin{pmatrix}
        -u_0 + 2u_1 - u2 \\
        -u_1 + 2u_2 + -u_3 \\
        -u_2 + 2 u_3 - u_4 \\
        \vdots \\
        -u_{N-2} + 2 u_{N - 1} - u_N
    \end{pmatrix}
    =
    \begin{pmatrix}
        f_1 \\
        f_2 \\
        f_3 \\
        \vdots \\
        f_{N-1}
    \end{pmatrix}
\end{align*}