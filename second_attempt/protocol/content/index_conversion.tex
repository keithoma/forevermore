\begin{formula}
    For the conversion from coordinates along the axis to the linear numbering of gird points and vice versa, we define
    \begin{align*}
        \idx_{d} &: \{1, \dots, n - 1\}^d \rightarrow \{1, \dots, N\} \text{, and} \\
        \idx_{d}^{-1} &: \{1, \dots, N\} \rightarrow \{1, \dots, n - 1\}^d \text{.}
    \end{align*}
    The exact way of computation is intuitive, yet tedious to formulate symbolically.

    \textit{Dimension 1}
    \begin{align*}
        \idx_{d = 1} (j_1) & = j_1 \\
        \idx_{d = 1}^{-1} (j_1) & = j_1 \\
    \end{align*}
    \textit{Dimension 2}
    \begin{align*}
        \idx_{d = 2} (j_1, j_2) & = (j_1 - 1) (n - 1) + j_2 \\
        \idx_{d = 2}^{-1} (N) & = (j_1, j_2) \text{, where}
        \begin{cases}
            j_1 = \ceil*{\frac{N}{n - 1}} \\
            j_2 = N - (j_1 - 1) (n - 1)
        \end{cases}
    \end{align*}
    \textit{Dimension 3}
    \begin{align*}
        \idx_{d = 3} (j_1, j_2, j_3) & = (j_1 - 1) (n - 1)^2 + j_2 (n - 1) + j_3 \\
        \idx_{d = 3}^{-1} (N) &= (j_1, j_2, j_3) \text{,} \\
        \text{where}&
        \begin{cases}
            j_1 = \ceil*{\frac{N}{(n - 1)^2}} \\
            j_2 = \ceil*{\frac{N - (j_1 - 1) (n - 1)^2}{n - 1}} \\
            j_3 = N - (j_2 - 1) (n - 1) - (j_1 - 1) (n - 1)^2
        \end{cases}
    \end{align*}

    \textit{Derivation} \hspace{0.1cm} The formula for \(d = 1\) should be clear. For \(d = 2\), given two coordinates \(j_1\) and \(j_2\), then one has \(j_1 - 1\) times of filled columns which has \(n - 1\) elements, thus we have \(N = (j_1 - 1) (n - 1) + j_2\). The inverse is slightly more difficult. To compute \(j_1\), we need to consider how many columns are filled by \(N\). This is done by \(\ceil*{\frac{N}{n-1}}\). Subtracting \((j_1 - 1)(n - 1)\) from \(N\) we get \(j_2\). The same ideas applies for \(d=3\).
    \begin{flushright}
        \(\bigtriangleup\)
    \end{flushright}
\end{formula}