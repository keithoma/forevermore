\subsection{To Infinity and Beyond}

We will generalize our understanding to two and three dimension. Let \([0, 1]^d\) the domain where \(d \in \{1, 2, 3\} \) is the number of dimensions. Given a discretization of the domain with \(n \in \mathbb{N}\) number of Cartesian grid on each axis, we have a total of \(N := (n-1)^d\) points inside the domain to consider. From the definition of the Laplace operator, we have
\begin{align*}
	\Delta u_N =& \frac{\partial^2 u_N}{\partial {x_1} + \frac{\partial^2 u_N}{\partial x_2^2} \text{.}
\end{align*}
We can approximate the partial abbreviation again with the Taylor polynomial. We have
\begin{align*}
	\frac{\partial^2 u_i}{\partial x_1^2} =& \frac{1}{h^2} (u_{i - 1} + 2u_i - u_{i + 1}) \\
	\frac{\partial^2 u_i}{\partial x_2^2} =& \frac{1}{h^2} (u_{i - (n-1)} + 2u_i - u_{i + (n +1)}) \text{.}
\end{align*}
Risking informality, but perhaps stimulating the intuition, one uses the neighboring "top and bottom" values to approximate the partial abbreviation for the direction of the x-axis while "left and right" values are used for the direction of the y-axis. The latter is exactly why we subtract and add \((n-1\) from the indices. 