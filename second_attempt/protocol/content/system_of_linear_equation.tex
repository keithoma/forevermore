\subsection{To Infinity and Beyond}

We will generalize our understanding to two and three dimension. Let \([0, 1]^d\) the domain where \(d \in \{1, 2, 3\} \) is the number of dimensions. Given a discretization of the domain with \(n \in \mathbb{N}\) number of Cartesian grid on each axis, we have a total of \(N := (n-1)^d\) points inside the domain to consider. We order every grid point as we discussed in the previous section and assign a positive integer to it. Let therefore \(p_i\) with \(1 \leq i \leq N\) the grid points on the domain. From the definition of the Laplace operator, we have
\begin{align*}
	-\Delta u_i =& -\sum^d_{l=1}\frac{\partial^2 u_i}{\partial x_l} \text{.}
\end{align*}
With the same idea as before, we want to approximate the partial derivative with the Taylor polynomial. We have
\begin{align*}
	-\frac{\partial^2 u_i}{\partial x_l} \approx \frac{1}{h^2}
	(-u(p_i - h e_l) + 2u(p_i) -u(p_i + h e_l)) \text{,}
\end{align*}
where \(e_l \in \mathbb{R}^{d}\) is the unit vector. Again, each grid point in every direction of the canonical basis is apart exactly \(h\) units. Therefore, we have for the first three dimensions (in case of \(d=2\), the third line below can be ignored)
\begin{align*}
	-\frac{\partial^2 u_i}{\partial x_1} \approx& \frac{1}{h^2}
	(-u_{i - 1} + 2u_i -u_{i + (n - 1)}) \\
	%
	-\frac{\partial^2 u_i}{\partial x_2} \approx& \frac{1}{h^2}
	(-u_{i - (n - 1)} + 2u_i -u_{i + (n - 1)}) \\
	%
	-\frac{\partial^2 u_i}{\partial x_3} \approx& \frac{1}{h^2}
	(-u_{i - (n - 1)^2} + 2u_i -u_{i + (n - 1)^2}) \text{.}
\end{align*}
Hence we have (again ignore the last summand for \(d=2\) below and simplify the equation accordingly)
\begin{align*}
	-\Delta u_i \approx \frac{1}{h^2} & ( (-u_{i - 1} + 2u_i -u_{i + (n - 1)} ) \\
	+&(-u_{i - (n - 1)} + 2u_i -u_{i + (n - 1)}) \\
	+&(-u_{i - (n - 1)^2} + 2u_i -u_{i + (n - 1)^2}) ) \\
	= \frac{1}{h^2} & 
	( -u_{i - (n - 1)^2} -u_{i - (n - 1)} -u_{i - 1} \\
	+& 6u_i \\
	+& u_{i + 1} + u_{i + (n - 1)} + u_{i + (n-1)^2} )
\end{align*}
Remember that if the index \(i\) is \(i \leq 0\) or \(i \geq N\), then \(u_i\) is on the boundary of the domain \(\Omega\) and we have \(u_i = 0\). With this, we can define the matrix \(A_d \in \mathbb{R}^{(n-1)^d \times (n-1)^d}\). We will do so recursively for convenience. Let
\begin{align*}
	A_1 (z) :=&
	\begin{pmatrix}
		2z & -1 & 0 & \dots & 0 \\
		-1 & 2z & -1 & & \vdots  \\
		0 & -1 & 2z & & \\
		\vdots & & &    & -1\\
		0 & \dots & &  -1 & 2z
	\end{pmatrix} \\
	\\
	A_d (z) :=&
	\begin{pmatrix}
		A_{d-1}(z) & -\mathcal{I}_{d-1} & 0 & \dots & 0 \\
		-\mathcal{I}_{d-1} & A_{d-1}(z) & -\mathcal{I}_{d-1} & & \vdots  \\
		0 & -\mathcal{I}_{d-1} & A_{d-1}(z) & & \\
		\vdots & & &    & -\mathcal{I}_{d-1}\\
		0 & \dots & &  -\mathcal{I}_{d-1} & A_{d-1}(z) \text{,}
	\end{pmatrix} \\
\end{align*}
where, \(\mathcal{I}_d \in \mathbb{R}^{(n-1)^d \times (n-1)^d}\) is the identity matrix. Then set \(A_d := A_d(d)\), and then we have
\begin{align*}
	\frac{1}{h^2} A_d \hat{u} = b \text{.}
\end{align*}