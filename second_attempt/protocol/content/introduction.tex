\section{Introduction}

This protocol discusses a numerical approach to solve a Laplace equation which for an \(u \in C^2(\mathbb{R^d}, \mathbb{R})\) is defined as
\begin{align*}
	\Delta u = \sum^{d}_{l = 1} \frac{\partial^2 u}{\partial x_l^2} \text{.}
\end{align*}
In particular, we will limit ourselves on the domain \( \Omega := (0, 1)^d \subset \mathbb{R^d} \) with the equations\footnote{\(\partial \Omega\) denotes the boundary of the domain.}
\begin{align*}
	- \Delta u &= f \hspace{0.5cm} \text{on } \Omega \text{, and} \\
	u &= 0 \hspace{0.5cm} \text{on } \partial \Omega \text{.}
\end{align*}
Given the function \(f\), how can one compute \(u\)? First, we must define a proper discretization of the domain and a convenient linear order for the grid points. This allows us to encode the values of the functions \(u\) and \(f\) for each grid point into a vector \(\hat{u}\) and \(b\). The notation for the vector \(\hat{u}\) is not just to differentiate from the function \(u\) we want to compute, but also to stress that the solution will be an approximation.

It turns out that we can actually construct a fairly simple linear transformation which maps the vector \(\hat{u}\) to \(b\). In other words, we will see that there is a matrix \(A\) such that
\begin{align*}
	A \hat{u} = b \text{.}
\end{align*}

Furthermore, this particular matrix \(A\), while large in size for almost all pratical applications, has a useful structure which we will exploit to reduce computation time to an acceptable length.

Lastly, as this protocol is meant to be preparatory piece, it will not present a method to solve the above linear equation for \(\hat{u}\).