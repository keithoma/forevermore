% Copyright 2019 Clara Eleonore Pavillet

% Author: Clara Eleonore Pavillet
% Description: This is an unofficial Oxford University Beamer Template I made from scratch. Feel free to use it, modify it, share it.
% Version: 1.0

\documentclass{beamer}
\usepackage[utf8]{inputenc}

\usepackage{amsmath}
\usepackage{mathtools}

\usepackage{xcolor}
\usepackage{graphicx}

\newcommand\myeq{\stackrel{\mathclap{\normalfont\mbox{??}}}{=}}


\title{Mathematical and Computational Modelling}
%\titlegraphic{\includegraphics[width=2cm]{Theme/Logos/OxfordLogoV1.png}}
\author{Christian Parpart, Kei Thoma}
\date{} %\today

\begin{document}

\begin{frame}
    \frametitle{Problemstellung}
    \small
    \begin{alignat*}{5}
           && p_1 &&     && && p_0 \\
        {\color{white}a +}&& 93  &&   {\color{white}b} &&{\color{white}=}&& 172 \\
        {\color{white}a +}&& 193 &&   {\color{white}b} &&{\color{white}=}&& 309 \\
        {\color{white}a +}&& 187 &&   {\color{white}b} &&{\color{white}=}&& 302 \\
        {\color{white}a +}&& 174 &&   {\color{white}b} &&{\color{white}=}&& 283 \\
        {\color{white}a +}&& 291 &&   {\color{white}b} &&{\color{white}=}&& 443 \\
        {\color{white}a +}&& 184 &&   {\color{white}b} &&{\color{white}=}&& 298 \\
        {\color{white}a +}&& 205 &&   {\color{white}b} &&{\color{white}=}&& 319 \\
        {\color{white}a +}&& 260 &&   {\color{white}b} &&{\color{white}=}&& 419 \\
        {\color{white}a +}&& 212 &&   {\color{white}b} &&{\color{white}=}&& 361 \\
        {\color{white}a +}&& 169 &&   {\color{white}b} &&{\color{white}=}&& 267 \\
        {\color{white}a +}&& 216 &&   {\color{white}b} &&{\color{white}=}&& 337 \\
        {\color{white}a +}&& 144 &&   {\color{white}b} &&{\color{white}=}&& 230
    \end{alignat*}
\end{frame}

\begin{frame}
    \centering
    \frametitle{Problemstellung}
    \includegraphics[width=\textwidth]{scatter.png}
\end{frame}

\begin{frame}
    \frametitle{Problemstellung}
    \small
    \begin{alignat*}{5}
           && p_1 &&     && && p_0 \\
        {\color{white}a +}&&                     93 &&   {\color{white}b} &&{\color{white}=}&&                    172 \\
        {\color{white}a +}&&                    193 &&   {\color{white}b} &&{\color{white}=}&&                    309 \\
        {\color{white}a +}&& {\color{lightgray}187} &&   {\color{white}b} &&{\color{white}=}&& {\color{lightgray}302} \\
        {\color{white}a +}&& {\color{lightgray}174} &&   {\color{white}b} &&{\color{white}=}&& {\color{lightgray}283} \\
        {\color{white}a +}&& {\color{lightgray}291} &&   {\color{white}b} &&{\color{white}=}&& {\color{lightgray}443} \\
        {\color{white}a +}&& {\color{lightgray}184} &&   {\color{white}b} &&{\color{white}=}&& {\color{lightgray}298} \\
        {\color{white}a +}&& {\color{lightgray}205} &&   {\color{white}b} &&{\color{white}=}&& {\color{lightgray}319} \\
        {\color{white}a +}&& {\color{lightgray}260} &&   {\color{white}b} &&{\color{white}=}&& {\color{lightgray}419} \\
        {\color{white}a +}&& {\color{lightgray}212} &&   {\color{white}b} &&{\color{white}=}&& {\color{lightgray}361} \\
        {\color{white}a +}&& {\color{lightgray}169} &&   {\color{white}b} &&{\color{white}=}&& {\color{lightgray}267} \\
        {\color{white}a +}&& {\color{lightgray}216} &&   {\color{white}b} &&{\color{white}=}&& {\color{lightgray}337} \\
        {\color{white}a +}&& {\color{lightgray}144} &&   {\color{white}b} &&{\color{white}=}&& {\color{lightgray}230}
    \end{alignat*}
\end{frame}

\begin{frame}
    \frametitle{Problemstellung}
    \small
    \begin{alignat*}{5}
           && p_1 &&     && && p_0 \\
        {\color{red}a} +&&                     93 &&   {\color{teal}b}    &&               =&&                    172 \\
        {\color{red}a} +&&                    193 &&   {\color{teal}b}    &&               =&&                    309 \\
        {\color{white}a +}&& {\color{lightgray}187} &&   {\color{white}b} &&{\color{white}=}&& {\color{lightgray}302} \\
        {\color{white}a +}&& {\color{lightgray}174} &&   {\color{white}b} &&{\color{white}=}&& {\color{lightgray}283} \\
        {\color{white}a +}&& {\color{lightgray}291} &&   {\color{white}b} &&{\color{white}=}&& {\color{lightgray}443} \\
        {\color{white}a +}&& {\color{lightgray}184} &&   {\color{white}b} &&{\color{white}=}&& {\color{lightgray}298} \\
        {\color{white}a +}&& {\color{lightgray}205} &&   {\color{white}b} &&{\color{white}=}&& {\color{lightgray}319} \\
        {\color{white}a +}&& {\color{lightgray}260} &&   {\color{white}b} &&{\color{white}=}&& {\color{lightgray}419} \\
        {\color{white}a +}&& {\color{lightgray}212} &&   {\color{white}b} &&{\color{white}=}&& {\color{lightgray}361} \\
        {\color{white}a +}&& {\color{lightgray}169} &&   {\color{white}b} &&{\color{white}=}&& {\color{lightgray}267} \\
        {\color{white}a +}&& {\color{lightgray}216} &&   {\color{white}b} &&{\color{white}=}&& {\color{lightgray}337} \\
        {\color{white}a +}&& {\color{lightgray}144} &&   {\color{white}b} &&{\color{white}=}&& {\color{lightgray}230}
    \end{alignat*}
\end{frame}

\begin{frame}
    \frametitle{Problemstellung}
    \small
    \begin{alignat*}{5}
           && p_1 &&     && && p_0 \\
        {\color{red}a} +&& 93  &&   {\color{teal}b} &&=&& 172 \\
        {\color{red}a} +&& 193 &&   {\color{teal}b} &&=&& 309 \\
        {\color{red}a} +&& 187 &&   {\color{teal}b} &&=&& 302 \\
        {\color{red}a} +&& 174 &&   {\color{teal}b} &&=&& 283 \\
        {\color{red}a} +&& 291 &&   {\color{teal}b} &&=&& 443 \\
        {\color{red}a} +&& 184 &&   {\color{teal}b} &&=&& 298 \\
        {\color{red}a} +&& 205 &&   {\color{teal}b} &&=&& 319 \\
        {\color{red}a} +&& 260 &&   {\color{teal}b} &&=&& 419 \\
        {\color{red}a} +&& 212 &&   {\color{teal}b} &&=&& 361 \\
        {\color{red}a} +&& 169 &&   {\color{teal}b} &&=&& 267 \\
        {\color{red}a} +&& 216 &&   {\color{teal}b} &&=&& 337 \\
        {\color{red}a} +&& 144 &&   {\color{teal}b} &&=&& 230
    \end{alignat*}
\end{frame}

\begin{frame}
    \frametitle{Problemstellung}
    \small
    \begin{alignat*}{5}
           && p_1 &&     && && p_0 \\
        1{\color{red}a} +&& 93  &&   {\color{teal}b} &&=&& 172 \\
        1{\color{red}a} +&& 193 &&   {\color{teal}b} &&=&& 309 \\
        1{\color{red}a} +&& 187 &&   {\color{teal}b} &&=&& 302 \\
        1{\color{red}a} +&& 174 &&   {\color{teal}b} &&=&& 283 \\
        1{\color{red}a} +&& 291 &&   {\color{teal}b} &&=&& 443 \\
        1{\color{red}a} +&& 184 &&   {\color{teal}b} &&=&& 298 \\
        1{\color{red}a} +&& 205 &&   {\color{teal}b} &&=&& 319 \\
        1{\color{red}a} +&& 260 &&   {\color{teal}b} &&=&& 419 \\
        1{\color{red}a} +&& 212 &&   {\color{teal}b} &&=&& 361 \\
        1{\color{red}a} +&& 169 &&   {\color{teal}b} &&=&& 267 \\
        1{\color{red}a} +&& 216 &&   {\color{teal}b} &&=&& 337 \\
        1{\color{red}a} +&& 144 &&   {\color{teal}b} &&=&& 230
    \end{alignat*}
\end{frame}

\begin{frame}
    \frametitle{Problemstellung}
    \small
    \begin{align*}
        \underbrace{
        \begin{pmatrix}
            1 &  93 \\
            1 & 193 \\
            1 & 187 \\
            1 & 174 \\
            1 & 291 \\
            1 & 184 \\
            1 & 205 \\
            1 & 260 \\
            1 & 212 \\
            1 & 169 \\
            1 & 216 \\
            1 & 144
        \end{pmatrix}
        }_A
        \underbrace{
        \begin{pmatrix}
            a \\ b
        \end{pmatrix}
        }_x
        =
        \underbrace{
        \begin{pmatrix}        
            172 \\
            309 \\
            302 \\
            283 \\
            443 \\
            298 \\
            319 \\
            419 \\
            361 \\
            267 \\
            337 \\
            230
        \end{pmatrix}
        }_v
    \end{align*}
\end{frame}

\begin{frame}
    \frametitle{Problemstellung}
    \small
    \begin{align*}
        \underbrace{
        \begin{pmatrix}
            1 &  93 \\
            1 & 193 \\
            1 & 187 \\
            1 & 174 \\
            1 & 291 \\
            1 & 184 \\
            1 & 205 \\
            1 & 260 \\
            1 & 212 \\
            1 & 169 \\
            1 & 216 \\
            1 & 144
        \end{pmatrix}
        }_A
        \underbrace{
        \begin{pmatrix}
            a \\ b
        \end{pmatrix}
        }_x
        {\color{red}\myeq}
        \underbrace{
        \begin{pmatrix}        
            172 \\
            309 \\
            302 \\
            283 \\
            443 \\
            298 \\
            319 \\
            419 \\
            361 \\
            267 \\
            337 \\
            230
        \end{pmatrix}
        }_v
    \end{align*}
\end{frame}

\begin{frame}
    \frametitle{Theorie}
    \begin{align*}
        ||Ax - b||_2
    \end{align*}
    {\color{white}Frage: Wie bestimmen wir \(x = (a, b)^T\), sodass die obige Gleichung minimal ist?}
    \begin{enumerate}
        \item {\color{white}Bestimme die QR-Zerlegung von \(A\). Also \(A = QR\).}
        \item {\color{white}Berechne \(z := Q^T v\). Definiere \(z_1\) als die ersten \(n\) Einträge (Breite von \(A\), hier \(n = 2\)) von \(z\).}
        \item {\color{white}Definiere \(R_1\) als die Matrix, die die ersten \(n \times n\) Einträge von \(R\) enthält.}
        \item {\color{white}Löse \(R_1 x = z_1\) durch Rückwärtseinsetzen.}
    \end{enumerate}
    {\color{white}Dann gilt: \(\text{min}||Ax - b||_2 = z_2\).}
\end{frame}

\begin{frame}
    \frametitle{Theorie}
    \begin{align*}
        ||Ax - b||_2
    \end{align*}
    Frage: Wie bestimmen wir \(x = (a, b)^T\), sodass die obige Gleichung minimal ist?
    \begin{enumerate}
        \item {\color{white}Bestimme die QR-Zerlegung von \(A\). Also \(A = QR\).}
        \item {\color{white}Berechne \(z := Q^T v\). Definiere \(z_1\) als die ersten \(n\) Einträge (Breite von \(A\), hier \(n = 2\)) von \(z\).}
        \item {\color{white}Definiere \(R_1\) als die Matrix, die die ersten \(n \times n\) Einträge von \(R\) enthält.}
        \item {\color{white}Löse \(R_1 x = z_1\) durch Rückwärtseinsetzen.}
    \end{enumerate}
    {\color{white}Dann gilt: \(\text{min}||Ax - b||_2 = z_2\).}
\end{frame}

\begin{frame}
    \frametitle{Theorie}
    \begin{align*}
        ||Ax - b||_2
    \end{align*}
    Frage: Wie bestimmen wir \(x = (a, b)^T\), sodass die obige Gleichung minimal ist?
    \begin{enumerate}
        \item Bestimme die QR-Zerlegung von \(A\). Also \(A = QR\).
        \item {\color{white}Berechne \(z := Q^T v\). Definiere \(z_1\) als die ersten \(n\) Einträge (Breite von \(A\), hier \(n = 2\)) von \(z\).}
        \item {\color{white}Definiere \(R_1\) als die Matrix, die die ersten \(n \times n\) Einträge von \(R\) enthält.}
        \item {\color{white}Löse \(R_1 x = z_1\) durch Rückwärtseinsetzen.}
    \end{enumerate}
    {\color{white}Dann gilt: \(\text{min}||Ax - b||_2 = z_2\).}
\end{frame}

\begin{frame}
    \frametitle{Theorie}
    \begin{align*}
        ||Ax - b||_2
    \end{align*}
    Frage: Wie bestimmen wir \(x = (a, b)^T\), sodass die obige Gleichung minimal ist?
    \begin{enumerate}
        \item Bestimme die QR-Zerlegung von \(A\). Also \(A = QR\).
        \item Berechne \(z := Q^T v\). Definiere \(z_1\) als die ersten \(n\) Einträge (Breite von \(A\), hier \(n = 2\)) von \(z\).
        \item {\color{white}Definiere \(R_1\) als die Matrix, die die ersten \(n \times n\) Einträge von \(R\) enthält.}
        \item {\color{white}Löse \(R_1 x = z_1\) durch Rückwärtseinsetzen.}
    \end{enumerate}
    {\color{white}Dann gilt: \(\text{min}||Ax - b||_2 = z_2\).}
\end{frame}

\begin{frame}
    \frametitle{Theorie}
    \begin{align*}
        ||Ax - b||_2
    \end{align*}
    Frage: Wie bestimmen wir \(x = (a, b)^T\), sodass die obige Gleichung minimal ist?
    \begin{enumerate}
        \item Bestimme die QR-Zerlegung von \(A\). Also \(A = QR\).
        \item Berechne \(z := Q^T v\). Definiere \(z_1\) als die ersten \(n\) Einträge (Breite von \(A\), hier \(n = 2\)) von \(z\).
        \item Definiere \(R_1\) als die Matrix, die die ersten \(n \times n\) Einträge von \(R\) enthält.
        \item {\color{white}Löse \(R_1 x = z_1\) durch Rückwärtseinsetzen.}
    \end{enumerate}
    {\color{white}Dann gilt: \(\text{min}||Ax - b||_2 = z_2\).}
\end{frame}

\begin{frame}
    \frametitle{Theorie}
    \begin{align*}
        ||Ax - b||_2
    \end{align*}
    Frage: Wie bestimmen wir \(x = (a, b)^T\), sodass die obige Gleichung minimal ist?
    \begin{enumerate}
        \item Bestimme die QR-Zerlegung von \(A\). Also \(A = QR\).
        \item Berechne \(z := Q^T v\). Definiere \(z_1\) als die ersten \(n\) Einträge (Breite von \(A\), hier \(n = 2\)) von \(z\).
        \item Definiere \(R_1\) als die Matrix, die die ersten \(n \times n\) Einträge von \(R\) enthält.
        \item Löse \(R_1 x = z_1\) durch Rückwärtseinsetzen.
    \end{enumerate}
    {\color{white}Dann gilt: \(\text{min}||Ax - b||_2 = z_2\).}
\end{frame}

\begin{frame}
    \frametitle{Theorie}
    \begin{align*}
        ||Ax - b||_2
    \end{align*}
    Frage: Wie bestimmen wir \(x = (a, b)^T\), sodass die obige Gleichung minimal ist?
    \begin{enumerate}
        \item Bestimme die QR-Zerlegung von \(A\). Also \(A = QR\).
        \item Berechne \(z := Q^T v\). Definiere \(z_1\) als die ersten \(n\) Einträge (Breite von \(A\), hier \(n = 2\)) von \(z\).
        \item Definiere \(R_1\) als die Matrix, die die ersten \(n \times n\) Einträge von \(R\) enthält.
        \item Löse \(R_1 x = z_1\) durch Rückwärtseinsetzen.
    \end{enumerate}
    Dann gilt: \(\text{min}||Ax - b||_2 = z_2\).
\end{frame}

\begin{frame}
    \frametitle{Theorie}
    \includegraphics[width=\textwidth]{plot.png}
\end{frame}

\begin{frame}
\frametitle{Implementierung und Experimente}

Mit den ursprünglichen Daten ...

\includegraphics[width=\textwidth]{plot.png}

Kondition: $A = 74.26$, $A^T A = 5514.75$

Residuum: 32.90 (ohne $p_2$), 32.09 (mit $p_2$)
\end{frame}

\begin{frame}
Hier wurde nur die ersten 6 Werte betrachtet ...

\frametitle{Implementierung und Experimente}
\includegraphics[width=\textwidth]{plot1.png}

Kondition: $A = 72.47$, $A^T A = 5252.05$

Residuum: 1.54 (ohne $p_2$), 1.26 (mit $p_2$)
\end{frame}

\begin{frame}
Hier wurden die Werte gerundet ...

\frametitle{Implementierung und Experimente}
\includegraphics[width=\textwidth]{plot2.png}

Kondition: $A = 55.74$, $A^T A = 3107.40$

Residuum: 42.28 (ohne $p_2$), 39.43 (mit $p_2$)
\end{frame}


\begin{frame}
\frametitle{Auswertung und Zusammenfassung}

\begin{itemize}


\item Je weniger Werte, desto kleiner das Residuum, aber das Ergebnis wird wahrscheinlich recht inakkurat.
\item Präzisere Inputwerte sind vorzuziehen.
\end{itemize}

\end{frame}

\begin{frame}
\frametitle{Literatur}
\begin{enumerate}

\item \textit{Numerische Lineare Algebra}, Prof. Dr. Caren Tischendorf

\end{enumerate}

\end{frame}


\end{document}