\documentclass[refman]{article}
\usepackage[utf8]{inputenc}
\usepackage[english]{babel}


%

\usepackage{colortbl}
\usepackage{epigraph}
\usepackage{fancyhdr}
\usepackage{graphicx}
\usepackage{hhline}
%\usepackage{biblatex}
\usepackage{wrapfig}

\usepackage[procnames]{listings}
\usepackage{longtable}
\usepackage{tikz}
\usepackage{subcaption}
\usepackage{xcolor}
\usepackage{wrapfig}
\usepackage[nottoc,numbib]{tocbibind}
%
%\usepackage[T1]{fontenc}
\usepackage{lmodern}

\usepackage{amsfonts}
\usepackage{amsmath}
\usepackage{amsthm}
\usepackage{mathtools}

\usepackage{geometry}
 \geometry{
 a4paper,
 left=30mm,
 top=30mm,
 }
\pagestyle{fancy}

\newcommand{\idx}{\text{idx}}

\DeclarePairedDelimiter\ceil{\lceil}{\rceil}
\DeclarePairedDelimiter\floor{\lfloor}{\rfloor}

\theoremstyle{definition}
\newtheorem*{formula}{Formula}

\usepackage[document]{ragged2e}

\begin{document}

\begin{titlepage}
	\centering
	%\includegraphics[width=0.15\textwidth]{graphics/huberlin_logo}\par\vspace{1cm}
	{\scshape\LARGE Humboldt University of Berlin \par}
	\vspace{1cm}
	{\scshape\Large Projektpraktikum \par}
	\vspace{1.5cm}
	{\huge\bfseries Documentation of qr.py \par}
	\vspace{2cm}
	{\Large\itshape Christian Parpart \& Kei Thoma \par}
	\vfill

	\vfill

% Bottom of the page
	{\large \today\par}
\end{titlepage}

\tableofcontents
\newpage
The following is the documentation for the module \texttt{qr.py}.

\section{qr(A)}
This function computes the QR-decomposition.
\subsection*{Arguments}
A (ndarray): Two dimensional ndarray.
\subsection*{Returns}
(ndarray): Two dimensional ndarray of the QR-decomposition, Q.

(ndarray): Two dimensional ndarray of the QR-decomposition, R.

\section{full\_rank(A)}
A boolean-function to check if the matrix has full column rank.
\subsection*{Arguments}
A (ndarray): Two dimensional ndarray.
\subsection*{Returns}
(boolean): True if A has full column rank and False otherwise.

\section{solve\_QR(A, b)}
Solves the equation Ax = b for x with the QR-decomposition.
\subsection*{Arguments}
A (ndarray): Two dimensional ndarray. Should have full column rank.

b (ndarray): One dimensional ndarray.
\subsection*{Returns}
(ndarray): One dimensional ndarray for x.

\section{norm(A, b)}

Computes the normed error induced by solving the linear equation with the function above.

\subsection*{Arguments}

A (ndarray): Two dimensional ndarray. Should have full column rank.

b (ndarray): One dimensional ndarray.

\subsection*{Returns}

(float): The normed difference between Ax and b.

\section{condition(A)}

Calculates the condition of $A$ and $A^T A$.

\subsection*{Arguments}

A (ndarray): Two dimensional ndarray.

\subsection*{Returns}

(float, float): The condition of $A$ and $A^T A$.

\section{input\_data(file\_name, indices=None)}

Takes a properly formated text file and builds a two dimensional ndarray. Additionally, a list of indices can be passed to return specific entries.

\subsection*{Arguments}

file\_name (String): The path and the name of the text file.
        
indices (list): List of indices.

\subsection*{Returns}

(ndarray): The entries of the matrix with the passed indices. If the argument indices were left out, this returns the entire matrix.

\section{draw(data)}

Draws the plot of the linear regression according to the data passed.

\subsection*{Arguments}

data (ndarray): Two dimensional ndarray.

\section{norm\_residuals(data)}

    Computes the residuals for the solution with and without $p_2$.

\subsection*{Arguments}

    data (ndarray): Two dimensional ndarray.

\subsection*{Returns}

(float, float): The residuals for Ax = b without taking p2 into consideration and with.

\section{main()}

The main-function to demonstrate the capabilities of the module.






\end{document}